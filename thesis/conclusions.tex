\section{Research Contributions}
This research developed methods for identifying, quantifying, and using the unintentional EM emanations from computing devices. These unintentional EM emanations were previously studied for security purposes, for example to study how EM emanations can potentially be used to extract secret keys in cryptography. We have demonstrated that it is viable to automatically identify and quantify EM emanations, and to use EM emanations to profile programs and detect unknown code. These techniques may present a good solution for embedded processors such those used in IoT devices. The research contributions of this work are:
\begin{enumerate}
\item SAVAT, a new metric that quantifies the side channel signal caused by differences in code execution at the instruction level~\cite{CALLAN2014,Callan2015EMC,Callan2015}. SAVAT can be used by computer programmers to quantify the effect of single instruction differences in programs, and can be used by hardware designers to pinpoint leaking circuits. We presented a practical methodology for measuring \SAVAT on real machines that uses specially designed benchmarks to generate a signal at a known alternation frequency where it can be isolated from the rest of the EM emanations from the device under test and can be measured reliably with inexpensive equipment. We also proved that the methodology does measure \SAVAT given a simplified yet realistic processor and emanations model. Finally, we measured SAVAT for the EM emanations side channel for a small set of instructions for laptops, desktops, and an FPGA-based processor demonstrating SAVAT's utility, reliability, and repeatability.
\item FASE, a method for finding amplitude modulated side channel signals in computing devices. FASE uses the SAVAT microbenchmarks to generate detectable spectral patterns in the sidebands of all the carrier signals that are AM-modulated by specific system activities~\cite{FASE_2015}. We also presented an algorithm to automatically process FASE spectra and calculate the frequencies of modulated carriers~\cite{wang2016}. To demonstrate FASE's effectiveness, we applied it to several computer systems and found activity-modulated signals generated by voltage regulators, memory refresh activity, and DRAM clocks. We confirmed that FASE correctly separates emanated signals that are affected by specific processor and/or memory activity from those that are not. FASE can be used to find which parts of a system leak information about some aspect of program activity. Once the source of the leak is found, the strength of modulated signals can be reduced and the modulation can be weakened, i.e. we can disrupt the connection between program behavior and the variations in activity that modulate such signals. Furthermore, other uses of EM emanations (such as profiling and malware detection) also require automatically finding and characterizing carriers modulated by system activities. 
\item ZOP, a system for zero-overhead profiling which is non-intrusive and requires no hardware modifications or support~\cite{zop}. In exchange for the ability to profile software without any overhead, ZOP makes a small sacrifice in accuracy ($>$ 94\% accurate compared to a technique based on instrumentation on the benchmarks tested), and requires a training phase. ZOP generates profiling information based on unintentional emanations alone. ZOP runs the to-be-profiled program with instrumentation over a set of training inputs while simultaneously recording the EM emanations. This allows us to match the execution of code fragments to the EM emanations they generate. To profile a program with zero software and hardware overhead, ZOP uses the waveforms from training, and their waveform-to-code mapping, to predict the execution path taken by the profiled run. Our experimental results show that ZOP can predict profiling information with greater than 94\% accuracy for the benchmarks considered in our evaluation.
\item A demonstration extending ZOP to detect unknown code at a distance of 3 meters. In order to detect unknown code at a distance, we first created a metric to summarize the accuracy of whole program path prediction by adapting the string edit distance metric to compare the actual and ZOP-predicted sequences of markers. Then we implemented ZOP on a PIC32 processor to determine how microarchitectural events affect ZOP's accuracy. Next we characterized the demodulated signals used by ZOP. We observed that the choice of antenna can greatly affect the demodulated signal. We showed that simple AM demodulation only captures a portion of the available information in the signal, particularly when the unintentionally modulated signals generated by computing devices contain multiple synchronous components with different phases (some modulated and some unmodulated). We also described a signal quality metric, and used this metric to quantify the effects distance and demodulation bandwidth have on signal quality. Finally, we demonstrated that ZOP can be adapted to detect the execution of unknown code in a known program with 92\% accuracy.
\end{enumerate}

\section{Future Research Directions}
The largest opportunity for future work is the improvement and extension of the ZOP approach to predicting the path through programs. This approach can be adapted to several applications, such as profiling, performance optimization, debugging, detecting malware, and the extraction of secret information from programs. With the ZOP approach, all of these applications can be implemented with zero hardware support, and with zero software and hardware direct hardware interaction while being used. There are several research directions to pursue to make ZOP feasible under more conditions. First, ZOP must be scaled to larger programs and different devices. The largest challenge with scaling ZOP to larger programs is input generation. ZOP's accuracy is strongly affected by how well the training input set covers the profiled program. For larger programs, it can be intractable to generate inputs automatically that cover the entire program and some method is needed to decompose the program into smaller pieces (via unit testing for example) which are small enough for automatic input generation. Furthermore, to enable this automatic input generation, we need a coverage metric which captures ZOP's marker-centric coverage requirements.

In addition to the future research possibilities for ZOP in the field of software engineering, there are many opportunities in signal processing. For example, better algorithms for time warping, tree search, using program structure/statistics to guide tree search, and waveform matching, as well better probing and noise cancellation will all improve ZOP's performance. Furthermore, running ZOP on more complex systems and on devices with operating systems presents other opportunites for further research. When a program is run on a device with an operating system, the program will be periodically interrupted and occasionally the processor will switch contexts and run a completely unrelated program. This switching needs to be tracked and dealt with by ZOP and this will require further research. In addition, more complex processors present additional signal processing challenges as it is likely that more bandwidth will be needed (resulting in the need for interference cancellation). High frequency processors also generally employ spread spectrum clocking to spread the processor and memory EM emanations over a wide range of frequencies. Demodulating these signals will require further processing to ``de-spread'' the processor EM emanations. 



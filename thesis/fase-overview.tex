This chapter describes FASE (Finding Amplitude-modulated Side-channel Emanations). FASE systematically and efficiently identifies periodic EM emanations whose amplitudes change as a result of specific changes in system activity, i.e. signals that are amplitude-modulated by system activity. Our methodology uses the SAVAT micro-benchmarks to generate repetitive changes in processor and memory activity, then processes the resulting EM signals to find spectral patterns corresponding to amplitude modulation. The EM spectrum is full of amplitude-modulated signals (e.g. radio broadcasts) that are not modulated by program activity. FASE filters out such signals by generating several different activity patterns and reporting only those signals which are specifically modulated by all the generated activity patterns. 

Side channels based on physical side-effects (power consumption, sound, or EM emanations) are difficult for microarchitects and programmers to alleviate, in part because the relationship between computational behavior and the resulting side channel signal is very complex and poorly understood. EM emanations side channels may be the most complex: the emanated signals may theoretically be anywhere in the EM spectrum, and signals at different frequencies may provide attackers with insight into different aspects of computational activity.  Therefore, the first step to use or mitigate side channel leakage is to identify signals that have some dependence on the secret information of interest. Much previous work addresses finding leakage signals as part of the process of carrying out an attack~\cite{Kocher99,gebotys2005,meynard2011,sugawara2009}. However, many of these side channel attack descriptions only briefly or implicitly address the underlying mechanisms that cause information leakage. While attacks do not require determining the cause of leakage, efficient mitigation does. Without a systematic approach to identification and causation, the process of finding root causes is time-consuming and mostly trial-and-error: the defender makes an educated guess about the leakage source, fixes the hypothesized problem, and sees whether the leakage has been reduced.

The FASE approach for discovering AM-modulated signals is highly effective at both finding modulated leakage sources, and at determining the type of activities causing the leakage. Computer systems generate thousands of periodic EM emanations. FASE successfully rejects all such signals that are not modulated by system activity, while reporting the small number of remaining signals that are modulated by specific system activities. This chapter describes unintentional AM modulated signals in computer systems, and then describes how FASE can be used to find and characterize such signals. To test FASE, we use it to find AM signals on a number of different computer systems. We also identify the source of each periodic signal and the mechanism by which it was modulated to demonstrate the usefulness of FASE and to understand potential EM side-channel vulnerabilities of modern processor and memory systems. Finally, we present a fully automated version of FASE and use it to find AM modulated signals on a large range of devices.

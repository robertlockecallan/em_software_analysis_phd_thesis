Efficient targeted mitigation of side-channel vulnerabilities requires finding information-leaking signals and determining how information is embedded into these signals. In this chapter we described FASE, a novel methodology for automatically finding which EM-emanated signals from a computer system are amplitude-modulated by specific program activities. FASE uses the SAVAT microbenchmarks to generate detectable spectral patterns in the side-bands of all the carrier signals that are AM-modulated by specific system activities, automatically processes measured spectra to identify these patterns, and calculates the frequencies of the modulated carriers.

This approach has several advantages. First, it directly identifies the carrier frequencies modulated \textit{by specific system activities}, which goes a long way toward determining the sources of compromising emanations. Second, it is robust against the interference of unmodulated signals and noise inside and outside of the system, such as AM-modulated signals and carrier-like signals which are not specifically modulated by system activity. Third, it quantifies how strongly carrier signals are modulated, which is useful for identifying how the carrier is generated, for quantifying information leakage, and for evaluating the effectiveness of mitigation efforts. Fourth, it is specifically designed to robustly detect unintentionally modulated signals, which have several inconvenient features not found in ideal AM signals. Finally, each FASE evaluation requires only a few spectrum measurements while other techniques such as DPA require thousands of spectrum captures with different keys and plaintexts~\cite{sugawara2009}. 

To demonstrate FASE's effectiveness, we applied it to several computer systems and found activity-modulated signals generated by voltage regulators, memory refresh activity, and DRAM clocks. Our results indicate that separate signals may carry different information about system activity, potentially enhancing an attacker's capability to extract sensitive information. We also confirm that our methodology correctly separates emanated signals that are affected by specific processor and/or memory activity from those that are not.

We also presented an algorithm for automatically measuring FASE. We demonstrated the algorithm's performance on several different types of processors and systems (desktops, laptops, and smartphones) and compared the results to an exhaustive manual search. We also verify that all signals identified by the algorithm can be traced to plausible unintentional modulation mechanisms to illustrate that these signals can potentially cause information leakage. 

FASE can be used to find which parts of a system leak information about some aspect of program activity. Once the source of the leak is found, the strength of modulated signals can be reduced and the modulation can be weakened, i.e. we can disrupt the connection between program behavior and the variations in activity that modulate such signals. Using memory refresh signals as an example, this would involve randomization of the interval between refresh commands, while modulation-weakening efforts might involve careful scheduling of memory accesses to avoid their interaction with refresh activity.



\section{Motivation}
Previous research has thoroughly studied how program activity on computing devices can leak sensitive information. At first such research was conducted only in secret~\cite{Khun03}, and then publicly to address the leakage of information from CRT displays~\cite{Eck85}, and next resurged again in the field of side channel cryptoanalysis~\cite{Kocher99}. Recent research has demonstrated that EM emanations leak information about a very wide range of system activities and that this leaked information might be useful for many new applications such as profiling, malware detection, and debugging. These new applications differ from the typical cryptoanalysis side channel attack scenario in several ways. First, system designers employ countermeasures against side channel attacks that weaken the signal, increase noise, and weaken the link between the emanations and leaked information. In these new applications, however, the monitored system is not hostile and so no countermeasures are present, making the extraction of useful information from EM emanations less difficult. Second, the structure of information needed for the new applications is more complex and varies from application to application and from problem instance to problem instance. Side channel attacks typically attempt to extract a secret key (a set of a few hundred bits which are used repeatedly to encrypt or decrypt data), whereas the new applications attempt to extract more complex information, such as the execution path through a program. Finally, the reward for demonstrating a successful side channel attack against a single device is relatively high. In comparison, the reward for demonstrating these new applications of EM emanations on a single device and single program is lower. 

These differences show that the new applications require a different approach. While an application specific and effort intensive approach makes sense for side channel attacks, the new applications can take advantage of stronger (unguarded) signals but also must systematically characterize hardware and software differences between problem instances and must automatically carry out many of the steps which could be done manually during side channel attacks. In order to be viable, analyses for these new applications must also be automatically applicable across a wider variety of devices, software types, and types of information to be extracted. These analyses therefore require systematic and automated identification, quantification, and usage of EM emanations. 

%Specifically, we need to automatically 1) identify the leakage caused by specific circuits and instructions using SAVAT, 2) find which signals are modulated by useful information using FASE, and 3) automatically determine the structure of a program to be profiled and make the connection between this structure and the observed EM emanations as is done during Zero Overhead Profiling and during malware detection.

\section{SAVAT: A Practical Methodology for Measuring the Side-Channel Signal Available to the Attacker for~Instruction~Level~Events}

Previous studies of the information embedded in EM emanations have focused almost exclusively on how emanations can be used to compromise a device's security. Specifically, EM emanations have been used in a variety of side channel attacks to circumvent traditional security protections and access controls in many different types of computing devices. Unlike traditional attacks that exploit vulnerabilities in what the system does, side channel attacks access information by observing how the system does it. Computation generates many types of electronic and microarchitectural activities. Side channel attacks identify some physical or microarchitectural signal (\ie the side channel signal) that leaks desired information about system activity or the data being processed, and then analyze that signal as the system operates. Much work has been done to prevent particular side channel attacks, either by severing the tie between sensitive information and the side channel signal, or by trying to make the signal more difficult to measure. As attacks are found system designers modify and improve systems to reduce and remove very specific types of information leakage (most commonly the leakage of cryptographic keys). This makes such work very application specific and focused on a perpetual cycle of developing attacks and defenses for increasingly specific vulnerabilities. With each iteration of attack and countermeasure, the information leakage becomes weaker or harder to extract, making the attacks and countermeasures increasingly sophisticated and application specific. Furthermore, the countermeasures are often applied \textit{after} the attack methods have been discovered. Other approaches to defending against side channel attacks include adding metal shielding and introducing large amounts of random electronic noise to the system. These approaches are typically applied globally to the whole system, making them expensive and power hungry. The technique we propose, SAVAT, differs from all these approaches because it is both proactive (i.e. can be used before an attack occurs) and allows fixes which can be targeted locally at the leaking circuitry or code. 
 %Because of countermeasures, existing methods used for side channel attacks are relatively effort intensive, requiring significant work to get the attack to work on a single device and single version of a single program or implementation. 

Information leakage can be quantified at many levels of granularity, ranging from differences in emanations across phases of a program's execution down to information leakage caused by specific hardware components such as transistors. However, in order to identify specific leaking circuits or parts of a computer program, a level of granularity is required that simultaneously exposes the contributions of both hardware and software, \ie the instruction level. SAVAT quantifies information leakage at the instruction level and develops benchmarks which can be used to quantify information leakage from specific instructions and system activities such as arithmetic operations or memory accesses. We also present measurements demonstrating the usefulness, reliability, and repeatability of SAVAT, as well as a theoretical model showing that SAVAT does measure values that can be used to quantify how single instruction differences affect side channel signals in the time domain. 

\section{FASE: Finding Amplitude-modulated Side-channel Emanations}

Information leakage in computing devices can be caused by many different system components and occurs across the EM spectrum, and leakage signals are obscured by noise created by other system components and signals from the external environment such as radio broadcasts, wireless communications, and power equipment. Many of the most useful leakage signals are generated by system components that generate strong periodic signals (carriers) which are modulated by the information of interest. In order to effectively use (or minimize) the information embedded in these modulated EM emanations, it is necessary to determine system activities that modulate these carriers, determine the frequency range and strength of the leaked signals, and determine the modulation mechanisms causing the leakage. FASE presents a method for finding existing computer system signals that are amplitude modulated by a specific type of system activity. We will also present measurements showing the types of signals FASE can find, and present an algorithm for automatically finding leakage signals using FASE.

\section{ZOP: Zero-Overhead Profiling via EM Emanations}

Applications that analyze software via EM emanations must be automatically applied to arbitrary computer programs running on computing devices. Zero-Overhead Profiling (ZOP) is one such application. ZOP uses EM emanations to generate path profiles for computer programs without using any instrumentation during profiling. A program profiler dynamically analyzes a program to collect statistics about the program's behavior. Path profiling counts the number of times a specific static path occurs in the execution of a program. This type of profiling is used to identify the most commonly executed paths (or regions) of a program. This information is very useful for code optimization and performance analysis. Path profiling is usually implemented by either adding instrumentation to the profiled program that counts the executions of each desired path as the program runs, or by using dedicated hardware features to record this information. 

Using instrumentation can provide perfectly accurate profiling information (i.e. the exact number of times a particular static path occurred), but the instrumentation code adds some runtime and space overhead to the original program which is undesirable. Runtime overhead can change the control flow of a program if that program interactions with the real world (e.g. has realtime deadlines, is part of a cyberphysical system, etc.). This makes profiling such systems challenging, especially when the goal is to observe the system ``in the field'' without disturbing it. ZOP, in contrast, uses zero instrumentation and requires no hardware features, making it especially desirable for these scenarios, and desirable in any scenario where overhead is unacceptable or undesirable. The tradeoff for zero overhead is that ZOP is not perfectly accurate, though ZOP's accuracy is high enough for most profiling usage cases. We show how ZOP uses a training phase to develop a model of how EM emanations can be related to program behavior, specifically how we can extract example EM waveforms that correspond to short sections of program execution by observing EM emanations while the program is running a set of training inputs, and how based on EM emanations alone we can systematically uses these training waveforms to predict a program profile over a separate set of program executions. We demonstrate ZOP on three small control-flow oriented benchmarks, showing that ZOP can profile control flow with high accuracy. We also characterize how training input coverage affects ZOP's performance. 

\section{Detection of Unknown Code on Internet of Things Devices at a Distance}

Detection of previously unseen malware is a challenging problem, particularly on Internet of Things devices. Such devices are vulnerable to malware because their functionality requires them to be connected to the internet. They are difficult to secure because they have limited hardware and software resources, diverse software and hardware environments are used in their development, and because updating such devices is difficult. IoT devices can be attractive for malicious purposes such as Distributed Denial of Service attacks because they are produced and deployed in large volumes. These properties make monitoring and verifying control via EM emanations attractive, particularly since there is an airgap between the monitor and the monitored device, making it impossible for an attacker to circumvent the protections even if all device's software is compromised. 

ZOP can be used to detect unseen malware by predicting the control flow through the program, while simultaneously keeping track of the confidence of its predictions over the course of the program. When the monitored system only runs known code, ZOP's prediction confidence will be high through the entire run of a program since the observed waveform behavior should match the training waveform behavior well. If, however, unknown code (e.g. malware) runs on the device, new program activity (and therefore new waveform behavior) will be observed, and ZOP's confidence in its predictions will drop. Therefore, we can predict the presence of malware by observing the confidence of ZOP's predictions.

This application will also require the monitoring device to be separated from the to-be-monitored devices so that numerous devices can be monitored by a single monitor, and so that the monitoring is unobtrusive. This work also presents some more detailed characterization of the EM emanations used by ZOP, specifically presenting a method for quantifying ZOP signal quality, and showing how antennas and distance affect the signals used.

\section{Research Contributions}
The research contributions of this thesis are
\begin{itemize}
\item SAVAT, a practical methodology for measuring the side-channel signal available to the attacker for instruction-level events~\cite{CALLAN2014}
\item A comparison of SAVAT values across laptops, desktops, and an FPGA-based processor~\cite{Callan2015EMC} 
\item Measurements demonstrating SAVAT's utility, reliability, repeatability, and validity~\cite{Callan2015}
\item FASE, a method for finding amplitude-modulated side channel EM emanations~\cite{FASE_2015}
\item An algorithm for automating FASE~\cite{wang2016}
\item ZOP, a method for path profiling computer programs with zero hardware and software overhead~\cite{zop}
\item A demonstration of detecting unknown code on an IoT device at a distance of 3 meters
\end{itemize}

\section{Thesis Outline}
The remainder of this thesis is organized as follows. Chapter~\ref{sec:literature_survey} describes previous and current related research and explains how this work relates to that research. Chapter~\ref{sec:savat} describes the SAVAT methodology for quantifying an individual instruction's contribution to side channel signals, demonstrates the usage of SAVAT on laptops, desktops and an FPGA-based processor, and shows SAVAT's reliability, repeatability, and theoretical validity. Chapter~\ref{sec:fase} describes FASE, a method for finding amplitude-modulated side channel EM emanations and develops an algorithm for automating FASE.
Chapter~\ref{sec:zop} describes ZOP, a method for path profiling computer programs with zero hardware and software overhead. Chapter~\ref{sec:malware_detect} presents more detailed characterization of the EM emanations used by ZOP and how they can be used to detect unknown code, specifically presenting a method for quantifying signal quality, and showing how antennas and distance affect the signals used. Finally, Chapter~\ref{conclusions} summarizes the thesis contributions and presents possible future directions for related research.

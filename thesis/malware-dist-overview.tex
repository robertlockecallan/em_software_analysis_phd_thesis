One of the emerging applications that makes use of the information embedded in EM emanations is the verification of control flow through a program, specifically the detection of previously unseen (i.e. zero day) malware. A rapidly growing number of embedded devices with internet connectivity are used in consumer electronics, Internet of Things devices, as well as industrial control and data acquisition applications. Securing these devices presents new and unique challenges due to their very limited software and hardware resources, the difficulty of applying software updates in the field, and the lack of standardization among the many hardware and software platforms used across devices. Techniques that make use of EM emanations to secure these devices may be a good fit for this application because such techniques do not require intrusive access or modification to the devices, and because remotely monitoring the devices in an air-gapped manner makes it impossible for an attacker to hide malicious activity by disabling security measures on the compromised device. Another advantage of using EM emanations to detect malware is that this detection can be done ``wirelessly,'' since it does not require a physical connection between the monitor and the monitored device. A single monitor could potentially observe the EM emanations from all the computing devices in a room and simultaneously secure all of them. Therefore we need to determine how strong the relevant EM emanations from the monitored devices are and how much distance can separate the monitor and the monitored devices while still accurately detecting malware. 

This chapter adapts ZOP for detecting malware, characterizes the effects which degrade ZOP's performance when monitoring devices from a distance, and shows results for detecting malware using ZOP at distance of 3 meters. Section~\ref{adapt_zop} describes how we can apply the algorithms developed for ZOP to detect malware. Section~\ref{malware-dist-static} examines the effect that microarchitectural events have on prediction accuracy by presenting a comparison of ZOP's whole program path prediction accuracy between NIOS and PIC32 processors. Section~\ref{malware-dist-snr} presents measurements that describe the effect on ZOP performance of using EM emanations at different clock harmonics, as well as the impact of using different antennas, signal bandwidths, and measurement distances, and defines a signal quality metric for ZOP. Section~\ref{malware-dist-detect} presents measurement results for detecting unknown code in executions of a known program at a distance of 3 meters, and Section~\ref{malware-dist-summary} summarizes this chapter. 



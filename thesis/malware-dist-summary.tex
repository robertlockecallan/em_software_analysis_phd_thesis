This chapter demonstrated that ZOP can be adapted to detect unknown code on an IoT device at a distance of 3 meters. Chapter~\ref{zop} presented ZOP, which counts the number of short acyclic intraprocedural paths executed during a computer program. For more general control flow verification, we needed a metric which gives a single value for each execution input indicating how much the predicted execution path differs from the actual execution path. For this purpose we adapted the string edit distance, using markers as string symbols. 

To adapt ZOP for detecting unknown code, we needed to identify the effects which can degrade accuracy, such as waveform variations in dynamic instances of a given static path (due to micro-architectural events) and signal effects such as demodulation, bandwidth, path loss, and antennas. To examine the effect of micro-architectural events, we implemented ZOP on a simpler PIC32 processor which did not have significant variations in dynamic behavior along a given static path. We compared ZOP results on NIOS vs PIC32 and found some sections of the tested benchmarks which have poor static path coverage, resulting in poor accuracy when these sections are encountered (on both PIC32 and NIOS). In addition, we found that the accuracy was measurably worse overall on NIOS, indicating that coverage of variations in dynamic behavior of a given static path (e.g. the effects of cache misses on NIOS) do significantly affect accuracy.

We next characterized the properties of the demodulated time domain signals used in ZOP, especially those properties which affect signal quality. We described how the demodulated signal varies with the antenna and harmonic used, and discovered that the presence of multiple synchronous clock components (modulated and unmodulated) can result in reduced signal levels when simple AM demodulation is used. We also described a metric for measuring the signal quality for demodulated signals generated by program execution. We used this metric to summarize the decay of signal strength as a function of the distance between the monitored device and the monitor, and presented measurements showing that for the execution of a given execution path, higher frequency components of the demodulated signal have greater variation, resulting in a reduced signal quality as more baseband bandwidth is used. Finally, we presented a measurement showing that ZOP can detect the presence of unknown code in a program execution with 92\% accuracy at a distance of 3 meters.




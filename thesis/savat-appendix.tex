\chapter{The Relationship between Side Channel Energy and Microbenchmark Spectral Power}

As discussed in Section~\ref{savat-metric}, we need to quantify the difference in energy available to an attacker between two signals $s_a(t)$ and $s_b(t)$ defined as
\begin{equation}
  \textrm{SAVAT}(s_a,s_b) \equiv \int_{0}^{T_s} (s_a(t) - s_b(t))^2 dt/R
\end{equation}
and relate this SAVAT to the spectral power, $P(f_{alt})$, observed while running each A/B alternation benchmark.

To simplify our analysis, we use the following model:
\begin{enumerate}
\item All processor instructions have execution time $T_I$.
\item $s_a(t)$ and $s_b(t)$ are voltages sampled at frequency $1/T_I$ to create the sequences $s_a[n]$ and $s_b[n]$ of length \hbox{$N_s = \frac{T_s}{T_I}$}.
\item The frequency content of $s_a(t)$ and $s_b(t)$ above $\frac{1}{2T_I}$ is negligible (i.e. $s_a(t)$ and $s_b(t)$ have bandwidth $\frac{1}{2T_I}$).
\item $s_a(t)$ and $s_b(t)$ are voltages measured across a resistance $R$.
\item The discrete time $\textrm{SAVAT}(s_a,s_b)$ is then
  \begin{equation}
    \textrm{SAVAT}(s_a,s_b) \equiv T_I \sum_{n=0}^{N_s-1} \frac{(s_a[n]-s_b[n])^2}{R}.
  \end{equation}
\item If the only difference between $s_a$ and $s_b$ is that instruction B is executed instead of instruction A at a single time sample $n_e$, then we define
  \begin{equation}
    \label{eqn:discrete_ese}
    \textrm{SAVAT}(A,B) \equiv \textrm{SAVAT}(s_a,s_b) = \frac{T_I}{R} (a_v - b_v)^2
  \end{equation}
  where $a_v = s_a[n_e]$ and $b_v = s_b[n_e]$.
\item For two $s_a(t)$ and $s_b(t)$ that differ by many instructions at many different time points, $\textrm{SAVAT}(s_a,s_b)$ can be calculated by adding all the $\textrm{SAVAT}(A,B)$ values for each instruction difference. In other words, a single $\textrm{SAVAT}(A,B)$ for each type of A and B instruction is sufficient to model $\textrm{SAVAT}(s_a,s_b)$ regardless of instruction ordering and the number of instructions changed between $s_a$ and $s_b$ (i.e. differences are additive and time invariant). 
\end{enumerate}

%As described previously, there are many challenges for measuring instruction SAVAT ($\textrm{SAVAT}(A,B)$) for complex processors with GHz clock frequencies.
The micro-benchmarks described in Section~\ref{savat-metric} create an alternation signal at frequency $f_{alt}$ by repeatedly executing instruction A $n_{inst}$ times, followed by $n_{inst}$ executions of instruction B. We then measure $P(f_{alt})$, the spectral power at frequency $f_{alt}$. To relate our measurements to the discrete time $\textrm{SAVAT}(A,B)$ defined above, we will show that 
\begin{equation}
  \label{eqn:meas_ese}
  \textrm{SAVAT}(A,B) \approx \Big(\frac{\pi}{2}\Big)^2 \frac{P(f_{alt}) \cdot N}{f_{alt} \cdot n_{inst}}.
\end{equation}

The signals generated by the SAVAT benchmarks can be represented as a specific mixture of two \textit{periodic} signals with period $N$. For $n=0,...,N-1$, the first signal is $a[n] = [o_0, o_1, ..., o_{N-2}, a_v]$. $a[n+N] = a[n]$ since $a[n]$ is periodic. The second signal is $b[n] = [o_0, o_1, ..., o_{N-2}, b_v]$. $a_v$ is the single sampled voltage at the time point where instruction A is active, $b_v$ is the sampled voltage at the time point where instruction B is active, and $o_n$ represent the other instructions in the benchmark necessary to make the benchmark practical (e.g. to create a loop around instruction A or instruction B).

\begin{comment}
  For the $n=0,...,N-1$ portion of $a[n]$ and $b[n]$, we see that 
  \begin{equation}
    \begin{aligned}
      \textrm{SAVAT}(a[n],b[n]) & = \frac{T_I}{R} \sum_{n=0}^{N} (a[n]-b[n])^2 \\
      & = \frac{T_I}{R} (a[N-1] - b[N-1])^2 \\
      & = \frac{T_I}{R} (a_v - b_v)^2 \\
      & = \textrm{SAVAT}(A,B).
    \end{aligned}
  \end{equation}
\end{comment}

To relate $a[n]$ and $b[n]$ to our benchmarks, we define a square wave $w[n]$ with a 50\% duty cycle such that
\begin{equation} \begin{aligned}
    & w[0  \leq n < N n_{inst}] = 1 \\
    & w[N n_{inst} \leq  n < 2N n_{inst}] = 0.
\end{aligned} \end{equation}

$w[n]$, $a[n]$, and $b[n]$ are then all periodic with period $2N n_{inst}$, and so we can take the discrete Fourier series of these signals over $2N n_{inst}$ samples. Call $A[k]$, $B[k]$, and $W[k]$ the discrete Fourier series (DFS) of $a[n]$, $b[n]$ and $w[n]$ respectively, defined for $0 \leq k < 2N n_{inst}$. Below we use several discrete Fourier series results presented in Appendix~\ref{DFS}.

We next define
\begin{equation}
  v[n]  = w[n] a[n] + (1-w[n]) b[n]
\end{equation}
which represents the signal created by the sequence of instructions executed by the microbenchmarks.

Observe that $V[k]$ (the DFS of $v[n]$) is
\begin{equation}
  \begin{aligned}
    V[k] & = W[k] \ast A[k] + (1-W[k]) \ast B[k] \\
         & = B[k] + W[k]\ast(A[k] - B[k])
  \end{aligned}
\end{equation}
where $\ast$ denotes periodic convolution, defined in Appendix~\ref{DFS}.

Now we consider $V[1]$, the $2^{\rm nd}$ Fourier coefficient (first harmonic) of the $v[n]$ sequence:
{
\small
\begin{equation}
  \label{eqn:V1}
  \begin{aligned}
      V[1] & = B[1] + \frac{1}{2N n_{inst}} \sum_{m=0}^{2N n_{inst}-1} W[1-m] (A[m] - B[m]) \\
      & = \frac{1}{2N n_{inst}} \sum_{l=0}^{N-1} W[1-2n_{inst}l] (A[2n_{inst}l] - B[2n_{inst}l])
  \end{aligned}
\end{equation}
}
The second equation follows since $A[k]$ and $B[k]$ are non-zero only for $k = 2n_{inst} l$ for \hbox{$l=0,1,...,N-1$} by Equation~\ref{eqn:periodic_dfs}.

\vskip 0.1in
$V[1]$ expands to
\begin{equation}
  \label{eqn:v_eq_rab}
  \begin{aligned}
    V[1] & = \frac{1}{2N n_{inst}} W[1] (A[0] - B[0]) \\
    & + \frac{1}{2N n_{inst}} W[1-2n_{inst}] (A[2n_{inst}] - B[2n_{inst}])  \\
    & + \ldots
  \end{aligned}
\end{equation}


The next few higher order odd harmonics are similar ($W[k] = 0$ for even $k$). For example, $V[3]$ expands to
\begin{equation}
  \begin{aligned}
    V[3] & = \frac{1}{2N n_{inst}} W[3] (A[0] - B[0]) \\
    & + \frac{1}{2N n_{inst}} W[3-2n_{inst}] (A[2n_{inst}] - B[2n_{inst}])  \\
    & + \ldots
  \end{aligned}
\end{equation}

To approximate $|V[1]|$, we assume $n_{inst}$ is large (e.g. $n_{inst} > 100$) which is true in practice. $W[k]$ is the $k^{\rm th}$ coefficient of the discrete Fourier series for a square wave with period $2Nn_{inst}$ where (by \cite{DTSP}, Example 8.3)
\begin{equation}
  \begin{aligned}
    \label{eqn:w1_approx}
    |W[k]| & = \frac{sin(\pi k/2)}{sin(\frac{\pi k}{2Nn_{inst}})} \\ 
    \frac{|W[k]|}{2Nn_{inst}} & = \frac{sin(\pi k/2)}{2Nn_{inst} \cdot sin(\frac{\pi k}{2Nn_{inst}})} \\
    \frac{|W[k]|}{2Nn_{inst}} & \approx \frac{sin(\pi k/2)}{\pi k} \\
    \frac{|W[1]|}{2Nn_{inst}} & \approx \frac{1}{\pi}.
  \end{aligned}
\end{equation}
The last two steps follow by recognizing that
\begin{equation}
  \begin{aligned}
    2Nn_{inst} \cdot sin \Big(\frac{\pi k}{2Nn_{inst}}\Big) = \pi k \cdot sinc\Big(\frac{\pi k}{2Nn_{inst}}\Big)
    \end{aligned}
\end{equation}
and noting that $sinc(x) \rightarrow 1$ as $x \rightarrow 0$ (i.e. large $n_{inst}$).

\vskip 0.1in
For $n_{inst} > 100$, $|W[1]| > 100 |W[1-n_{inst}]|$, so the higher order terms in Equation~\ref{eqn:v_eq_rab} can be ignored, giving
\begin{equation}
  \begin{aligned}
    \label{eqn:v1_ab_approx}
    |V[1]| & \approx \frac{|W[1]|}{2N n_{inst}}  \cdot |A[0] - B[0]| \\
    \pi |V[1]| & \approx |A[0] - B[0]|.
  \end{aligned}
\end{equation}
%Also note that the frequency content of A and B are very similar, especially for the first few n since a and b differ only by one sample. Also, recall that many A[k] and B[k] are zero.

\vskip 0.1in
We will next show that $A[0] - B[0] = 2 n_{inst} (a_v - b_v)$. 
To see this, we decompose $a[n] = o[n] + a_d[n]$ where the first $N$ samples of $o[n] = [o_0, o_1, ..., o_{N-2}, 0]$ and the first $N$ samples of $a_d[n] = [0, ..., 0, a_v]$. We can decompose $b[n]$ similarly.  By the linearity of the Fourier transform
\begin{equation}
   \begin{aligned}
     A[k] - B[k] & = A_d[k] + O[k] - ( B_d[k] + O[k] ) \\
      & = A_d[k] - B_d[k].
   \end{aligned}
\end{equation}
The DFS coefficient $A_d[0]$ is
\begin{equation}
  \begin{aligned}
  A_d[0] & = \sum_{n=0}^{2Nn_{inst}-1} a_d[n]
         & = 2n_{inst} a_v.
  \end{aligned}
\end{equation}
Similarly $B_d[0] = 2n_{inst} b_v$. Therefore
\begin{equation}
  \label{eqn:avbv}
  A[0] - B[0] = A_d[0] - B_d[0] =  2n_{inst} (a_v - b_v).
\end{equation}

Combining Equations~\ref{eqn:v1_ab_approx} and \ref{eqn:avbv},
\begin{equation}
  \label{eqn:avbv_V1}
  \begin{aligned}
    2n_{inst} |a_v - b_v| & \approx \pi |V[1]| \\
    |a_v - b_v| & \approx  \frac{\pi |V[1]|}{2n_{inst}}.
  \end{aligned}
\end{equation}

Also note that
\begin{equation}
  \label{eqn:inst_per_sec}
  \begin{aligned}
    n_{inst} f_{alt}  & = \frac{1}{2NT_I}\\
    T_I & = \frac{1}{2Nn_{inst} f_{alt}} 
  \end{aligned}
\end{equation}
since both sides of the first equation equal the number of A or B instructions executed per second. 

The power observed with the spectrum analyzer is described by (\cite{Heinzel2002,NumericalC})
\begin{equation}
  \label{eqn:P_f_alt}
  \begin{aligned}
    P(f_{alt}) & = \frac{2}{R} \Big(\frac{ |V[1]|}{2Nn_{inst}}\Big)^2 \\
    R \cdot 2N^2(n_{inst})^2 \cdot P(f_{alt}) & = |V[1]|^2.
  \end{aligned}
\end{equation}

\vskip 0.1in
Using Equations \ref{eqn:discrete_ese}, \ref{eqn:avbv_V1}, \ref{eqn:inst_per_sec}, and \ref{eqn:P_f_alt}:
\begin{equation}
  \label{eqn:ese_pwr}
  \begin{aligned}
    \textrm{SAVAT}(A,B) & = \frac{T_I}{R} (a_v-b_v)^2
    \hfill & \text{(Eq. \ref{eqn:discrete_ese})}\\
    & \approx \frac{T_I}{R} \pi^2 \frac{|V[1]|^2}{(2n_{inst})^2}
    %
    \hfill & \text{(by Eq. \ref{eqn:avbv_V1})}\\
    & \approx \Big(\frac{\pi}{2}\Big)^2 \frac{T_I}{R}  \frac{|V[1]|^2}{(n_{inst})^2} \\
    & \approx \Big(\frac{\pi}{2}\Big)^2 \frac{1}{R\cdot 2Nn_{inst} \cdot f_{alt}} \frac{|V[1]|^2}{(n_{inst})^2}
    \hfill & \text{(by Eq. \ref{eqn:inst_per_sec})}\\
    %
    & \approx \Big(\frac{\pi}{2}\Big)^2 \frac{R\cdot 2N^2(n_{inst})^2 \cdot P(f_{alt})}{R\cdot 2Nn_{inst} \cdot f_{alt}} \frac{1}{(n_{inst})^2}
    \hfill & \text{(by Eq. \ref{eqn:P_f_alt})} \\
    %
    & \approx \Big(\frac{\pi}{2}\Big)^2 \frac{P(f_{alt}) \cdot N}{f_{alt} \cdot n_{inst}}
    \hfill & \text{(Eq. \ref{eqn:meas_ese})} \\
  \end{aligned}
\end{equation}
This shows that our model \hbox{$\textrm{SAVAT}(A,B) = \frac{T_I}{R} (a_v-b_v)^2$} is closely approximated by our measured SAVAT. In other words our hardware measurements record $P(f_{alt})$,
% $= \frac{|V[1]|^2}{R\cdot 2Nn_{inst}}$
the power at $f_{alt}$ (the fundamental frequency of $v[n]$), and we convert to $\textrm{SAVAT}(A,B)$ using the above equation.

\chapter{Discrete Fourier Series}
\label{DFS}
The discrete Fourier series as defined in \cite{DTSP},~Equations~8.11 and 8.12, for a periodic signal $x[n]$ with period $M$ is
\begin{equation}
  \begin{aligned}
    x[n]  & \xrightarrow{\rm DFS} X[k] \\
    X[k]  & = \sum_{n=0}^{M-1} x[n] e^{-j(2\pi/M)kn} \\
    x[n]  & = \frac{1}{M} \sum_{n=0}^{M-1} X[k] e^{-j(2\pi/M)kn}.
  \end{aligned}
\end{equation}

Multiplying two sequences $x_1[n]$ and $x_2[n]$ is equivalent to periodic convolution in the frequency domain %\hbox{($X_1[k] \ast X_2[k]$)}
as described in \cite{DTSP},~Section~8.2.5: 
\begin{equation}
  \begin{aligned}
    x_1[n] x_2[n] & \xrightarrow{\rm DFS} X_1[k] \ast X_2[k] \\
    X_1[k] \ast X_2[k] & = \frac{1}{M} \sum_{l=0}^{M-1} X_1[l]X_2[k-l]
  \end{aligned}
\end{equation}


One other discrete Fourier series result is needed. Consider a signal $x[n]$ which is periodic with period $L$, so that $x[n+L] = x[n]$ for all $n$. The discrete Fourier series of $x[n]$ taken \textit{over a period of $ML$ samples} is
\begin{equation}
  \label{eqn:periodic_dfs}
  \begin{aligned}
    X[k] &= \sum_{n=0}^{ML-1} x[n] e^{-j \frac{2\pi}{ML} kn} \\
    &= \sum_{l=0}^{L-1} \Big(x[l] \sum_{m=0}^{M-1} e^{-j \frac{2\pi}{ML} k (l + Lm)}\Big) \\
    &= \sum_{l=0}^{L-1} \Big(x[l] e^{-j \frac{2\pi}{ML}{kl}} \sum_{m=0}^{M-1} e^{-j 2\pi k \frac{m}{M}}\Big) \\
    &= \sum_{l=0}^{L-1} \Big(x[l] e^{-j \frac{2\pi}{ML}{kl}} \sum_{m=-\infty}^{\infty} \delta[k-Mm]\Big) \\
    &= \left\{
    \def\arraystretch{1.2}%
    \begin{array}{@{}c@{\quad}l@{}}
      \sum_{l=0}^{L-1} x[l] e^{-j \frac{2\pi}{ML}{kl}} & \text{for $k = Mm$}\\
      0 & \text{for $k \neq Mm$}\\
    \end{array}\right.
  \end{aligned}
\end{equation}

The fourth step follows from the equivalence of two forms of the discrete time periodic impulse train with period $M$ (\cite{DTSP},~Example~8.1):
\begin{equation}
  \label{eqn:impulse_train}
  \sum_{m=-\infty}^{\infty} \delta[k-Mm] = \frac{1}{M} \sum_{m=0}^{M-1} e^{-j 2\pi k \frac{m}{M}}. 
\end{equation}

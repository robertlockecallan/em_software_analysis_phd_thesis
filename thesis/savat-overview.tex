This chapter presents a metric, the Side-Channel Signal Available to the Attacker (SAVAT,~\cite{CALLAN2014}), which does not present or imply a specific side-channel attack, but instead provides direct quantitative feedback to programmers and hardware designers about which instructions (or combination of instructions) have the greatest potential to create side-channel vulnerabilities. For this purpose, it is best to analyze information leakage due to instruction execution because analyzing emanations at the circuit level (e.g. wires, transistors, and gates) does not address the effects of system architecture and software, and because analyzing emanations at the program or program phase level~\cite{Demme_SVF_ISCA12,Demme_SVF_TopPicks12} does not provide direct feedback to pinpoint leakage sources. SAVAT overcomes the difficulties in measuring information leakage in complex systems by generating controlled EM emanations to isolate the differences between instructions one pair at a time, and then measuring and analyzing these emanations in the frequency domain.

SAVAT measures the side channel signal created by a specific single-instruction difference in program execution. In other words, \SAVAT quantifies the signal made available to a potential attacker who wishes to decide whether the program has executed instruction/event A or instruction/event B. This level of granularity is neither too fine-grained nor too coarse, and therefore is useful to both computer architects (\SAVAT tells which microarchitectural features create strong leakage signals) and to software developers (\SAVAT allows them to systematically aggregate the leakages caused by single instruction differences throughout a program).

We also measure EM side-channel energy among several common instructions from a laptop, a desktop, and an FPGA at several different frequencies. We show that the SAVAT measurements performed at different frequencies result in comparable SAVAT values, up to a frequency dependent scale factor. We also confirm several expectations. First, the SAVAT values for a given instruction pair are much smaller on the FPGA compared to personal computers, as might be expected based on differences in power and performance levels in the systems. Second, between instruction pairs we observe similar trends across all devices. By comparing results between different systems, vulnerabilities that are consistent across several processor generations and among manufacturers can be determined, allowing designers and programmers to focus on the most endemic vulnerabilities.

To summarize, this chapter presents: 
\begin{enumerate}
\item SAVAT, a new metric that quantifies the side channel signal caused by differences in code execution at the instruction level, 
\item A practical methodology for measuring \SAVAT on real machines, 
\item A derivation proving that the methodology does measure \SAVAT given a simplified yet realistic processor and emanations model, and
\item SAVAT measurements for the EM emanations side channel for a small set of instructions for laptops, desktops, and an FPGA-based processor demonstrating SAVAT's utility, reliability, and repeatability.
%\item Simulations, measurements, and analysis comparing EM emanations \SAVAT to power \SAVAT for a NIOS processor. 
\end{enumerate}

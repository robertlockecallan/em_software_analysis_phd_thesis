
%Side channels enable a powerful class of attacks that circumvent traditional protections, and a significant number of such attacks and potential countermeasures have been proposed for both hardware and software. Recent work has shown that potential side channel vulnerability can be assessed at the level of an entire processor or system, and at the granularity of entire phases. However, without a practical way to attribute potential side channel vulnerability to specific instruction-level behavior, it is difficult for computer architects and software developers to apply countermeasures strategically to limit their cost and performance/power impact.

This chapter presented a new metric, which we call \SAVATfull (SAVAT), that measures the side channel signal created by a specific single-instruction difference in program execution, i.e. the amount of signal made available to a potential attacker who wishes to decide whether the program has executed instruction/event A or instruction/event B. We also devised a practical methodology for measuring SAVAT in real systems using only user-level access permissions and realistic measurement equipment. While similar metrics rely on time domain measurements, SAVAT is measured in the frequency domain, overcoming some challenges posed by time domain measurements of EM emanations caused by instruction execution in high performance systems. 

We measured SAVAT among several common x86 instructions on three different laptops and one desktop at several different frequencies. Our results showed that two systems with the same design have nearly identical measured SAVAT values, which implies that SAVAT measurements on one system are representative of an entire manufacturing run, or possibly an entire family, of systems. Our SAVAT measurements were precise (st.dev/mean $<$ 5\%) for each tested system. We also demonstrated that SAVAT measurements are consistent regardless of instruction order and other implementation details. We also measured the effect of unequal A and B instruction counts and showed that with appropriate normalization, SAVAT is consistent over a range of frequencies. % and we showed that the sources of EM side-channel emanations in computer systems can be modelled as a combination of Hertzian and magnetic dipoles and showed how SAVAT decays as a function of distance.
Finally, to illustrate the validity of SAVAT we derived a relationship between SAVAT and a simple time domain metric. 



Overall, we confirmed that our new metric and methodology can help discover the highest-vulnerability aspects of a processor architecture or a program, and thus inform decision-making about how to best manage the overall side channel vulnerability of a processor, program, or system. SAVAT can be used by circuit designers and microarchitects to reduce susceptibility to side channel attacks by focusing on high-SAVAT aspects of their designs (e.g. off-chip memory accesses, last-level-cache hits, and possibly the integer divider in the systems we measured). Programmers, compilers, and algorithm designers can also use SAVAT to guide code changes to avoid using ``loud'' activity when operating on sensitive data. Overall, our instruction-level metric and methodology differ from prior work in that they {\em quantify} the signal that is sent to the attacker by an {\em instruction-level difference} in program execution. These measurements can be used to determine the potential for information leakage when execution of individual instructions or even sections of code depends on sensitive information. We expect our instruction-level attribution of potential side channel vulnerability to help system designers decide {\em where in the system/processor} to apply countermeasures, and also to help programmers and compilers apply software-based countermeasures selectively to minimize their performance and power impact.
